\documentclass[12pt,a4paper]{article}
\usepackage[utf8]{inputenc}
\usepackage[left=1.5cm,right=1.5cm,
top=1.5cm,bottom=1.5cm,bindingoffset=0cm]{geometry}
\usepackage[english,russian]{babel}
\usepackage[pdftex]{graphicx}
\usepackage{amsfonts}
\usepackage{amsmath}
\usepackage{verbatim}
\begin{document}
	\begin{titlepage}
	
		\centerline{\large \bf Белорусский государственный университет}
		\centerline{\large \bf Факультет прикладной математики и информатики}
		\vfill
		\vfill
		\centerline{\Large \bf Методы простой итерации и Ньютона}
		\centerline{\Large \bf решения нелинейного уравнения}
		\bigskip
		\vfill
		\bigskip
		\vfill
		\begin{centering}
			{\large
				Отчет по лабораторной работе \\
				студентки 2 курса 3 группы \\
				Бурак Ирины \\
			}
		\end{centering}
		\vfill
		\vfill
		\hfill
		\begin{minipage}{0.25\textwidth}
			{\large{\bf Преподаватель} \\
				{Будник А.М.}}
		\end{minipage}
		\vfill
		\vfill
		\centerline{\Large \bf Минск 2015}
	\end{titlepage}
	\section{Постановка задачи}
	Дано нелинейное уравнение, записанное в виде $f(x) = 0$. Требуется найти решение этого уравнения.
	\section{Метод простой итерации}
	Для решения уравнения требуется отделить его корни. Пусть в области $|x^0 - x| \le \delta $, где $x^0$ -- начальное приближение, находится ровно один корень данного уравнения. Тогда исходное уравнение приводят к виду $x = \phi(x)$, где $\phi(x)$ -- некоторая функция, удовлетворяющая следующим условиям:
	\begin{enumerate}
		\item $\phi(x)$ определена на $|x^0 - x| \le \delta $, имеет там непрерывную производную $|\phi'(x)| \le q < 1$
		\item для начального приближения верно неравенство $|x^0 - \phi(x^0)| \le m$
		\item числа $m$, $q$, $\delta$ удовлетворяют неравенству $\frac{m}{1 - q} < \delta$
	\end{enumerate}
	В этом случае метод простой итерации $x^{k+1} = \phi(x^k)$ сходится к точному решению уравнения.
	\subsubsection*{Опеределение констант}
	Для уравнения $f(x) = 2^x + 5x - 3$ на отрезке $[0, 1]$ существует единственное решение, так как данная функция монотонна, а $f(0)*f(1) = (-3)*4 < 0$. Таким образом можно считать, что $\delta = 1/2$. \\
	Функцию $\phi(x)$ будем рассматривать в виде $\phi(x) = x + cf(x)$. При $c = -1/5$ получаем:
	\begin{itemize}
		\item $\phi(x) = \frac{1}{5}(3 - 2^x)$
		\item $|\phi'(x)| = |-\frac{1}{5}2^x\ln(2))| \le \frac{2\ln(2)}{5} < q = 1/3 < 1$
		\item при $x_0 = 1/2$ имеем: $|x^0 - \phi(x_0)| = |1/2 - \frac{1}{5}(3 - \sqrt{2})| < m = 1/5$
		\item $\frac{m}{1 - q} = 3/10 < \delta$
	\end{itemize}
	\section{Метод Ньютона}
	Итерационный метод Ньютона имеет вид $x^{k+1} = x^k - \frac{f(x^{k})}{f'(x^k)}$. \\ 
	В отличие от \textit{метода простой итерации} данный метод не требует приведения исходного уравнения к виду, удобному для итераций.
	\section{Листинг}
	В обоих методах итерационный процесс продолжался до тех пор, пока выполнялось неравенство $|x^{k+1} - x^k| < 10^{-5}$.
	\verbatiminput{Listing.txt}
	\section{Результаты}
	\includegraphics[scale = 0.8]{Results.png} \\ \\
	Метод простой итерации и Ньютона дали один и тот же ответ, однако метод Ньютона сошёлся быстрее и дал меньшую погрешность.
\end{document}